Introduction, pas eu de tâche particulière intégration au sein de l'équipe avec diff tâches (ajout fonctionnalités, correction bug, Documentation en aglais, refacto, optimisation, formation

\chapter{Gestion de projet agile}
\section{Méthodes agiles}
\paragraph{}
Le projet a été mené selon une approche agile. La gestion de projet agile repose sur un développement itératif qui consiste à découper un projet en plusieurs itérations appelé sprint. Les méthodes agiles définis plusieurs valeurs fondamentales : 
\begin{itemize}
\item La communication au sein de l'équipe
\item La collaboration avec le client, celui-ci doit être impliqué tous au long du développement.
\item L'adaptation au changement, c'est-à-dire que le planification initiale doit être flexible pour permettre l'évolution de la demande. 
\end{itemize}
Il existe différentes type de méthode agile (Scrum, XP, RAD...) qui possèdent leurs propres caractéristiques.
\paragraph{}
Pour le développement d'Open Orchestra, c'est la méthode Scrum qui est utilisé. Scrum définis trois rôles au sein d'une équipe : 
\begin{itemize}
\item  Le \og Product Owner (PO) \fg{} qui porte la vision du produit
\item Le \og Scrum Master \fg{} est le responsable de la mise en œuvre de la méthode, il doit s'assurer que cette dernière est correctement appliqué
\item L'équipe de développement qui est chargé de transformer les besoins exprimés par le Product Owner. 
\end{itemize}
\paragraph{}
La réalisation d'un projet utilisant la méthode est rythmée par différents évènements : 
\paragraph{}
Tous d'abord, le sprint qui est représente une période de courte durée, de une à quatre semaines, durant laquelle l'équipe effectue un nombre de tâches définis à l'avance.
 \paragraph{}
 La réunion de planification où l'équipe répond à deux questions, \og Quoi ? \fg{} c'est-à-dire les tâches du backlog\footnote{Le backlog est un ensemble de fonctionnalités ou de tâches nécessaire pour la réalisation satisfaisante d'un projet} qu'elle réalisera au prochain sprint et \og Comment ? \fg{} c'est a ce moment que l'équipe estime les tâches choisis au \og Quoi \fg{} 

 \paragraph{}
Le \og daily scrum \fg{} qui est une réunion réalisé quotidiennement. Durant cette réunion chaque membre de l'équipe indique les tâches qu'il a réalisé depuis le dernier daily et celles qu'il va effectuer jusqu'au prochaine daily et pour finir les difficultés qu'il a ou pense rencontrer.
Cette réunion permet à tous les membres de connaître les tâches de chacun afin de s'entraider et d'anticiper plus facilement les obstacles.
 \paragraph{}
Pour finir à la fin du sprint, les membres de l'équipe se réunisse pour la revue de sprint durant laquelle les différentes tâches réalisés durant le sprint sont présentés et validés. Puis pour la rétrospective qui permet de mettre en évidence les points positif et les actions a mettre en place pour améliorer le prochain sprint.

SCHEMA (schéma sprint)
\section{Scrum avec Open Orchestra}
MIse en place au sein de l'équipe (Trello, cérémonie, dayly)
détaille des colonnes trello 
Céremonie toutes les semaines,
Daily tous les midis (indique ce que l'on a fait et que ce compte faire permet de savoir qui fait quoi en permancence, 
prévenir si l'on voit un problème)
Workflow d'un ticket (colonnes backlog , TODO, Doing, ToValidate => Faile, Done)
\chapter{Intégration continue}
Git (Worflow, tag, travis) 
Utilisation de Git , présentatation rapide de git et github
Systémé de PR avec validation par l'équipe. 
Pour eviter au maximun les régréssion utilsiation de travis description plus tard)
Plate forme d'integration (la ou valide le PO), un déploiement plusieurs fois par jour sur l'environnement prod) utililisation capistrano ( présentation de l'outil) 
Deploiement (capistrano)
\chapter{Qualité du code}
PSR2 (qu'est ce que c'est, présentation)
code climate (outil qui vérifie la qualité du code et fournis une note) 
SensioInsignt (Outil de qualite d'application Symfony)
Travis, (Détail a quoi ça sert, comment cela fonctionne (hook github lancement des test) 
Test unitraite, Test fonctionel, (Mise en place de test  pour tous les codes metiers) 
Behat (mise en place de test behat) 
