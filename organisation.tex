\chapter{Organisation du développement}
Introduction, pas eu de tâche particulière intégration au sein de l'équipe avec diff tâches (ajout fonctionnalités, correction bug, Documentation en aglais, refacto, optimisation, formation

\section{Gestion de projet agile}
\subsection{Méthodes agiles}
\paragraph{}
Le projet a été mené selon une approche agile. La gestion de projet agile repose sur un développement itératif qui consiste à découper un projet en plusieurs itérations appelé sprint. Les méthodes agiles définis plusieurs valeurs fondamentales : 
\begin{itemize}
\item La communication au sein de l'équipe
\item La collaboration avec le client, celui-ci doit être impliqué tous au long du développement.
\item L'adaptation au changement, c'est-à-dire que le planification initiale doit être flexible pour permettre l'évolution de la demande. 
\end{itemize}
Il existe différentes type de méthode agile (Scrum, XP, RAD...) qui possèdent leurs propres caractéristiques.
\paragraph{}
Pour le développement d'Open Orchestra, c'est la méthode Scrum qui est utilisé. Scrum définis trois rôles au sein d'une équipe : 
\begin{itemize}
\item  Le \og Product Owner (PO) \fg{} qui porte la vision du produit
\item Le \og Scrum Master \fg{} est le responsable de la mise en œuvre de la méthode, il doit s'assurer que cette dernière est correctement appliqué
\item L'équipe de développement qui est chargé de transformer les besoins exprimés par le Product Owner. 
\end{itemize}
\paragraph{}
La réalisation d'un projet utilisant la méthode est rythmée par différents évènements : 
\paragraph{}
Tous d'abord, le sprint qui est représente une période de courte durée, de une à quatre semaines, durant laquelle l'équipe effectue un nombre de tâches définis à l'avance.
 \paragraph{}
 La réunion de planification où l'équipe répond à deux questions, \og Quoi ? \fg{} c'est-à-dire les tâches du backlog\footnote{Le backlog est un ensemble de fonctionnalités ou de tâches nécessaire pour la réalisation satisfaisante d'un projet} qu'elle réalisera au prochain sprint et \og Comment ? \fg{} c'est a ce moment que l'équipe estime les tâches choisis au \og Quoi \fg{} 

 \paragraph{}
Le \og daily scrum \fg{} qui est une réunion réalisé quotidiennement. Durant cette réunion chaque membre de l'équipe indique les tâches qu'il a réalisé depuis le dernier daily et celles qu'il va effectuer jusqu'au prochaine daily et pour finir les difficultés qu'il a ou pense rencontrer.
Cette réunion permet à tous les membres de connaître les tâches de chacun afin de s'entraider et d'anticiper plus facilement les obstacles.
 \paragraph{}
Pour finir à la fin du sprint, les membres de l'équipe se réunisse pour la revue de sprint durant laquelle les différentes tâches réalisés durant le sprint sont présentés et validés. Puis pour la rétrospective qui permet de mettre en évidence les points positif et les actions a mettre en place pour améliorer le prochain sprint.

SCHEMA (schéma sprint)
\subsection{Scrum avec Open Orchestra}
Au sein du projet Open Orchestra, la méthode Scrum est appliqué.. Ainsi des sprint de une semaine avec un daily scrum tous les jours a midi.
De plus, tous les mercredis ou mardis selon les disponibilités une \og cérémonie \fg{} qui comprend la réunion de planification et la rétrospective.
Concernant, la revue de sprint (validation des taches du sprint) celle-ci n'est pas faite durant une réunion définis mais tout au long du sprint par le product ownver.
 \paragraph{}
 Pour faciliter la mise en place de la méthode Scrum, l'équipe utilise différent outils.
 Tous d'abord Skype et Slack\footnote{Slack est une plateforme de communication réalisé en 2014 qui permet d'intégrer facilement différents outils de service en ligne tel que github, trello, dropbox, google drive, ...)} pour la communication.
  \paragraph{}
 Et enfin Trello pour organiser les tâches, en effet l'équipe possédé un \verb?board? avec différentes colonnes pour organiser les tâches : 
 \begin{itemize}
 \item[]
 \item \textbf{Backlog} : le backlog qui contient les différentes tâches a effectué pour le projet, le tâches du backlog sont organisées en quatre colonnes (Nice to have, Good, Great, Must have) selon la priorité de la tâche.
 \item \textbf{Support} : Les demandes effectué par les différentes équipes d'intégrateur d'Open Orchestra au sein d'Interakting.
 \item \textbf{Proposition} : Les différentes propositions d'amélioration ou de re- factorisation du code faites par les membres de l'équipe.
 \item \textbf{Bugs} : Les différents bug rencontrés durant le développement
 \item \textbf{Todo} : Les taches a effectuer durant le sprint en cours.
 \item \textbf{Doing} : Les taches actuellement en développement.
 \item \textbf{Blocked} : Les taches bloqué pour différentes raison (manque de précision de la tache, bug empêchant la réalisation de la taches)
 \item \textbf{ToDeploy} : Taches réalisés qui sont prêtes a être déployé sur le serveur d'intégration.
 \item \textbf{ToValidate} : Taches du sprint en attente de validation par le product owner.
 \item \textbf{Failed} : Taches du sprint e non validé par le product owner
 \item \textbf{Done} : Taches du sprint validé par le product owner
 \item[]
 \end{itemize}
Les colonnes \verb?ToDeploy?, \verb?Done? sont unique a un sprint, il y a donc une colonne  différent\verb?ToDeploy?, \verb?Done? pour chaque sprint.
\paragraph{}
Par exemple, une tâches suit un processus, présenté par le schéma, bien particulier entre l'intrégration de la tache dans la sprint et ca validation par le product ownver. Ce processus permet de suivre facilement l'avancement ou encore les différents problèmes rencontrés sur les différentes tâches par tous les membres de l'équipe.
\section{Intégration continue}
Pour le dévellopement d'Open Orchestra les principes de l'intégration continue sont utilisés. 
\begin{quotation}
L'intégration continue est un ensemble de pratiques utilisées en génie logiciel consistant à vérifier à chaque modification de code source que le résultat des modifications ne produit pas de régression dans l'application développée.
\end{quotation}
L'utilisation de ces pratiques apporte de nombreux avantages, tous d'abord comme il est indiqué dans la citation cela permet de minimiser les régréssions mais aussi d'avoir a tous moment un produit (logiciel, site, etc) utilisable.
\paragraph{}
La mise en place de cette technique nécéssite différentes pré-requis : 
\begin{itemize}
\item[]
\item Dépot unique de code source versionné
\item Automatisation des tests
\item Un environnement similaire a celui de production
\item Automatiser le deploiement
\end{itemize}

\subsection{git}
Dans le cadre d'Open Orchestra, le code source est versionné avec git\footnode{Git est un lgociel de versionning décentralisé, c'est-à-dire qu'il n'existe pas un dépôt unique mais un dépot local pour chaque développeurs.} et utilise GitHub qui est un service web permettant d'heberger un dépôt git. Comme je l'expliquais lors des présentations des caractèristiques d'Open Orchestra, ce dernier est découpé en plusieurs bundles et donc chaque bundles posséde son propre dépôt sur GitHub.
\paragraph{}
Open orchestra étant toujours en version béta, il n'y a pas de realease a proprement parlé. Ainsi le worklow de dévellopement sur github est simplifié car il n'y a qu'une seule branche stable, la branche maste. Les autres branches sont des branches temporaires d'ajout de fonctionnalités ou de correction de bugs. Ce type de workflow est appellé \verb?GitHub flow? et est représenté par le schéma.
\paragraph{}
Ainsi avec ce type de workflow, la modification du code source que ce soit pour corriger un bug ou ajouter une fonctionnalités s'effectue toujours de la même manière : 
\begin{enumerate}
\item[]
\item Mettre à jour sa branche master en local
\item Créer une nouvelle branche en local à partir de master
\item Effectuer les modifications dans cette branche
\item Pousser sa branche sur GitHub (origin)
\item Vérifier que les tests fonctionnne, je reviendrais plus en détail sur ce point par la suite
\item Ouvrir une pull-request, sur GitHub une pull-request consiste à faire une proposition de modification, à partir de cette pull-request les différents membres peuvent commenter, valider ou non la modification.
\end{itemize}

Une fois la pull-request validé, elle peut être mergé avec master, c'est-à-dire inclure la modification dans la branche master.
\paragraph{}
Lorsqu'une branche est poussé sur l'un des dépôts GitHub du projet, les tests sont automatiquement lancé. Pour cela, les différents dépôts du projet intégre un outil \verb?Travis CI?. Travais CI 

Pour eviter au maximun les régréssion utilsiation de travis description plus tard)
Plate forme d'integration (la ou valide le PO), un déploiement plusieurs fois par jour sur l'environnement prod) utililisation capistrano ( présentation de l'outil) 
Deploiement (capistrano)
\section{Qualité}
PSR2 (qu'est ce que c'est, présentation)
code climate (outil qui vérifie la qualité du code et fournis une note) 
SensioInsignt (Outil de qualite d'application Symfony)
Travis, (Détail a quoi ça sert, comment cela fonctionne (hook github lancement des test) 
Test unitraite, Test fonctionel, (Mise en place de test  pour tous les codes metiers) 
Behat (mise en place de test behat) 
