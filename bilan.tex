\chapter*{Bilan}
\addcontentsline{toc}{chapter}{Bilan}
Durant le stage, ce fut un réel plaisir de travailler avec l'équipe du projet Open Orchestra et plus en général avec les différentes personnes d'Interakting que j'ai côtoyé à l'agence de Caen ou de Paris lors de mes déplacements.
\paragraph{}
J'ai été directement intégré au projet comme aurait pu l'être un autre développeur de l'agence ce qui a été très constructif pour moi et important de voir que l'on me faisait confiance.
\paragraph{}
Travailler sur un projet Open Source d'un point de vue plus orienté éditeur que intégrateur (réalisation de site pour des clients) a été très intéressant et motivant.  Cela m'a permi de voir et mettre en pratique la méthode agile Scrum sur un projet concret.
\paragraph{}
D'un point de vue technique, j'ai pu comprendre et approfondir des concepts avancés de Symfony2. Plus généralement, j'ai appris énormément sur les bonnes pratiques (test, psr2, principe SOLID \footnote{Les cinq principes de bases de la programmation orientée objet :
\begin{itemize}
\item Single responsibility : Une classe doit avoir une seule responsabilité
\item Open/closed : Une classe doit être ouverte à l'extension, mais fermé à la modification
\item Liskov Substitution : Une instance de type A doit pouvoir être remplacé par une instance de type B tel que B implémente A
\item Interface segregation : Préférer plusieurs interfaces spécifiques plutôt qu'une seule interface générale
\item Dependency Inversion : Il ne faut pas dépendre des implémentations, mais des abstractions
\end{itemize}
}
) mais aussi de découvrir et d'utiliser de nouveaux outils/langages (Travis, Ansible, Vagrant, CoffeeScript, Jmeter). 
De plus, durant mon stage j'ai assisté à des formations sur différent sujet (Git, Travis, Scrutinizer) proposé tous les jeudis de 13h à 15h par des développeurs de l'entreprise.
\paragraph{}
Pour conclure, ce stage à largement répondu à mes attentes que ce soit sur le projet en lui-même ou sur les points techniques.