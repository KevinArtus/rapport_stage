\chapter{Le projet}
\section{Présentation}
        \paragraph{}
        Open Orchestra est un CMS (Content Management  System) open source en développement maintenant un peu plus d'un an. 
        \paragraph{}
        Il est basé sur le framework PHP MVC Symfony 2 et MongoDB. Open Orchestra offre bien-sûr les fonctionnalités attendus par tous CMS  : gestion de contenu, médiathèque, contribution de page, gestion de version, utilisateurs, \og worklow \fg{}, rôles,  multi-site, multi- langue,  multi-device , etc.
        \paragraph{}
        La grande particularités d'Open Orchestra est son faible couplage du code mais aussi au niveau des solutions techniques, c'est-à-dire que l'on peut facilement remplacer ou étendre un de ses composants. Je reviendrais plus en détail sur ce point dans une section ultérieur.
        
        \section{Architecture}
        Une des particularités d'Open Orchestra est que le \og Back-office \fg{} et le \og Front office \fg{} peuvent être dissociés dans deux applications Symfony différente. 

        Ce découpage a deux nombreux avantages, tous d'abord un seul \og Back-office \fg{}  peux gérer plusieurs sites (\og Front-office \fg{}), de plus cela permet de mettre \og Back-office \fg{} et le \og Front-office \fg{} sur des serveurs différents, la seule condition est qu'ils doivent utiliser la même base de données.
        \paragraph{}
		Pour finir, Open Orchestra utilise une API RESTFull\footnote{REST (Representational State Transfer) est un style d'architecture qui définie différentes réglés (ressource identifié unitairement, utilisation des verbes HTTP POST, DELETE, PUT, GET) pour accéder et manipuler des ressources } pour  accéder, gérer ces différentes entités (pages, utilisateurs, contenus, etc).
	    \paragraph{}	
		Le schéma présente l'architecture d'Open Orchestra avec un \og Back-office \fg{} qui gère plusieurs \og Fronts \fg{}.
   
\chapter{Fonctionnalités}
         \section{Content type}
          \section{Nodes}  
          \section{Zones}  
          \section{Blocks}  
\chapter{Caractéristiques}
   \section{Performance}
   cache esi
   varnish 
   \section{Modularités}
   decouplage bundles
   possibilité de désactivé bundle
        
        
        