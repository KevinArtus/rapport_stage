\chapter{Le projet}
\section{Présentation}
        \paragraph{}
        Open Orchestra est un CMS (Content Management  System) open source en développement maintenant un peu plus d'un an. 
        \paragraph{}
        Il est basé sur le framework PHP MVC Symfony 2. En compléments des composants attendus par tous CMS tel que la gestion de contenu, une médiathèque ou encore la contribution de page Open Orchestra offre des fonctionnalités avancés comme la gestion multi-site et multi-langue mais aussi multi-device.
        \paragraph{}
        De plus, l'une des grandes particularités d'Open Orchestra et sa modularités, c'est-à-dire que l'on peut facilement remplacer ou étendre un de ses composants.
        \section{Architecture}
        plusieurs front pour un back 
        back api 
        mongo 
        schéma

   
\chapter{Fonctionnalités}
         \section{Content type}
          \section{Nodes}  
          \section{Zones}  
          \section{Blocks}  
\chapter{Caractéristiques}
   \section{Performance}
   cache esi
   varnish 
   \section{Modularités}
   decouplage bundles
   possibilité de désactivé bundle
        
        
        