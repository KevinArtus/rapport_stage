\chapter{Le projet}
\section{Présentation}
        \paragraph{}
        Open Orchestra est un CMS (Content Management  System) open source en développement maintenant un peu plus d'un an. Il est actuellement en version bêta, la sortie de la première version est prévus pour le mois de septembre.
        \paragraph{}
        Il est basé sur le framework PHP MVC Symfony 2 et MongoDB. Open Orchestra offre bien-sûr les fonctionnalités attendus par tous CMS  : gestion de contenu, médiathèque, contribution de page, gestion de version, utilisateurs, \og worklow \fg{}, rôles,  multi-site, multi- langue,  multi-device , etc.
        \paragraph{}
        La grande particularités d'Open Orchestra est son faible couplage du code mais aussi au niveau des solutions techniques, c'est-à-dire que l'on peut facilement remplacer ou étendre un de ses composants. Je reviendrais plus en détail sur ce point dans une section ultérieur.
        
        \section{Architecture}
        Une des particularités d'Open Orchestra est que le \og Back-office \fg{} et le \og Front office \fg{} peuvent être dissociés dans deux applications Symfony différente. 

        Ce découpage a deux nombreux avantages, tous d'abord un seul \og Back-office \fg{}  peux gérer plusieurs sites (\og Front-office \fg{}), de plus cela permet de mettre \og Back-office \fg{} et le \og Front-office \fg{} sur des serveurs différents, la seule condition est qu'ils doivent utiliser la même base de données.
        \paragraph{}
		Pour finir, Open Orchestra utilise une API RESTFull\footnote{REST (Representational State Transfer) est un style d'architecture qui définie différentes réglés (ressource identifié unitairement, utilisation des verbes HTTP POST, DELETE, PUT, GET) pour accéder et manipuler des ressources } pour  accéder, gérer ces différentes entités (pages, utilisateurs, contenus, etc).
	    \paragraph{}	
		Le schéma présente l'architecture d'Open Orchestra avec un \og Back-office \fg{} qui gère plusieurs \og Fronts \fg{}.
   
\chapter{Fonctionnalités}
	    \paragraph{}
	    Dans ce chapitre je vais vous présenter les fonctionnalités d'Open Orchestra. Bien sur, je ne vais pas détailler tous les fonctionnalités juste les points clés pour une meilleur compréhension du projet.
	      \section{Blocs}  
	       \paragraph{}
	      Dans Open Orchestra tous les éléments visible en front sont représenté par des blocs. Enfaite un bloc est simplement une entité avec un certains nombre d'attribut qui varient selon le type de block, chaque type de bloc sont indépendant, c'est à dire que eux seul connaissent leurs attributs, leurs façon se d'affiche en front ou en back-office ou encore la façon dont on peut les contribuer en back-office. 
	       \paragraph{}
	       Pour permettre cette indépendance, on utilise le design pattern stratégie. Le pattern stratégie permet de rendre une famille d'algorithmes interchangeables et ainsi la possibilité d'exécuter un traitement spécifique selon le contexte.

          \paragraph{}
	       Comme le présente le schéma  pour l'affichage en front d'un bloc, chaque bloc possède une classe qui indique comment doit s'afficher le bloc pour cela elle a deux méthode, la première qui indique le bloc qu'elle supporte (\verb?support?) et la méthode \verb?show? qui fournis le rendu, d'un autre coté il y a une classe que l'on appelle \verb?manager?  qui est la seule a connaitre toutes les stratégies des différents bloc. 

         \paragraph{}
	       Ainsi, lorsque que l'on désire afficher un bloc, il suffit de demander au manager d'afficher ce bloc, ce dernier va chercher parmi les stratégies qu'il connait laquelle supporte (méthode \verb?support? le bloc et s'il en trouve une alors il exécute la méthode show de la stratégie.
	        
	      	\paragraph{}
	      	Par défaut, le CMS propose de nombreux type blocs  comme par exemple un bloc pour liste un type de contenu, un bloc pour afficher un texte quelconque formaté, pour afficher un carrousel ou un média, un menu, une carte, un bloc de contact, etc.
	      \paragraph{}
	      	 Bien-sûr, il possible pour un intégrateur \footnote{Dans le cadre de ce rapport un intégrateur indique une personne ou une équipe qui utilise Open Orchestra et l'étende pour des besoins particulier}, de créer son propre type de bloc, il lui suffit de développer les différentes stratégies (affichage en front et back-office, formulaire en back-office, etc) nécessaire a un bloc.  
         \section{Nodes}
         \paragraph{}
         Un des points central d'un CMS est les pages. Sur Open Orchestra les pages sont identifiés comme des noeuds (\og nodes \fg{}).
          Les nodes sont simplement des conteneurs qui contient des zones. Les zones permettent d'organiser la pages, elles contiennent des sous-zones ou des blocs ce qui permet un découpage fin de la page pour simplifier sont organisation. 
         \paragraph{}
         Il existe trois types de nodes : 
         \begin{itemize}
         \item[]
         \item  Les nodes qui représentent les pages visibles en \og front \fg{}.
          \item[]
         \item  Les \og node tranverse \fg{} qui contiennent les blocs transverses, c'est-à-dire les blocs qui sont communs à plusieurs pages comme le bloc \og header \fg{} ou encore le bloc \og footer \fg{}.
          \item[]
         \item Le dernier type de nodes sont ceux qui permettent de contribuer les pages d'erreurs (404, 503) d'un site.
         \end{itemize}
         \paragraph{}
		Le schéma montre un exemple typique de page avec trois zones, le header qui contient un bloc menu, le footer un bloc footer et une zone principale découpé elle-même en deux zones avec un bloc de contact pour la zone de droite et un bloc de texte à gauche   
         \section{Content type}
         Un autre aspect important d'un CMS est les contenus avec Open Orchestra il est facile de créer des types de contenus (news, client, etc, ...).
          \paragraph{}
          En effet, Open Orchestra propose une approche graphique grâce a un formulaire pour créer ou éditer des types de contenus comme l'illustre la figure.
          Un type de contenus est composé de différent champs avec différent option, par exemple pour un champ de type texte il peut y avoir une option pour limiter le nombre de caractère ou encore une valeur par défaut.
          \paragraph{}
          Par défaut, Open Orchestra offre quelques types de champs (date, texte, monnaie, média, email, entier, zone de texte riche, etc). Comme pour tous les composants d'Open Orchestra il est possible pour les intégrateurs d'ajouter d 'autre type de champ avec ces options.  
\chapter{Caractéristiques}
   \section{Performance}
   Open Orchestra a était développé pour supporter de grosse charge de trafic. Pour cela le CMS exploite au niveau du front l'ESI (Edge Side Includes) couplé a un reverse proxy\footnote{Un reverse proxy est un serveur qui traite les requêtes en amont du serveur web. L'intéret d'un rervese proxy est multiple gestion de cache, chiffrement, répartition de chage} tel que Varnish.
   \paragraph{}
   L'ESI est un langage de balisage HTML qui permet de diviser une page en différent éléments dont les rendus est fait dans différentes requêtes par le serveur web. Ce qui permet d'avoir un cache HTTP sur les différents éléments et ainsi rafraichir seulement les éléments obsolètes de la page et non toute la page.
   \paragraph{}
   Comme nous l'avons vue dans la section sur Open Orchestra les pages sont déjà découpé en différent blocs, l'utilisation de l'ESI permet nettement l'amélioration des performances.

   Si nous reprenons l'exemple de notre page (Numéro), le schéma présente le processus effectué par le serveur dans le cas où les blocs \verb?header? et \verb?footer? sont déjà en cache lorsqu'un utilisateur demande la page.

   http://naholyr.fr/2011/03/varnish-edge-side-includes-cache-partiel/
   
   \section{Modularités}
   Comme je vous l'explicais en introduction un des particularités d'Open Orchestra est sa modularité. Pour cela, les différents composants du CMS sont répartis dans différents bundles \footnote{Sous Symfony2 un bundle est un ensemble de fichiers structurés(contrôleur, entité commande, listener, formulaire) qui permettent d'implémenter une ou des fonctionnalités qui dans l'idéale peut être réutilisé dans différent projets}
   \paragraph{}
   \begin{itemize}
   \item \textbf{open-orchestra-base-bundle} : Classes communes au back-office et front-office
   \item \textbf{open-orchestra-base-api-bundle} : Api d'Open Orchestra
   \item \textbf{open-orchestra-base-api-mongo-model-bundle} : Implémentation des models nécessaire à l'api pour \verb?MongoDB? 
   \item \textbf{open-orchestra-cms-bundle} : Logique du back-office
   \item \textbf{open-orchestra-front-bundle} : Logique du front-office
   \item \textbf{open-orchestra-display-bundle} : Logique d'affichage des blocs en Front-office
   \item \textbf{open-orchestra-media-bundle} : Médiathéque d'Open Orchestra
   \item \textbf{open-orchestra-media-admin-bundle} : Administration de la médiathéque d'Open Orhcestra
   \item \textbf{open-orchestra-model-interface} : Interface des models utilisé par les autres bundles 
   \item \textbf{open-orchestra-model-bundle} : Implémentation des models pour \verb?MongoDB? 
   \item \textbf{open-orchestra-user-bundle} : Gestion des utilisateurs 
   \item \textbf{open-orchestra-workflow-function-bundle} : Système de fonction worklow pour la gestion en back-office 
   \end{itemize}
   \paragraph{}
   Ainsi ce découpage fin a de nombreux avantages. Une application front-office n'a aucun intérét à charger toute la logique du back-office ainsi grâce a ce découpage chaque application (front-office ou back-office) charge uniquement les composant qui lui sont nécéssaire.
   \paragraph{}
   De plus, cela permet de désactiver facilement une fonctionnalités. Par exemple, si un intégrateur n'a pas bessoin de gérer des médias il peut alors désactiver la médiathèque en n'utilisant par les bundles (\verb?open-orchestra-media-bundle? et \verb?open-orchestra-media-admin-bundle?).

   \paragraph{}
   Un autre avantage est la facilité de remplacement d'un composants. Par défaut Open Orchestra utilise \verb?MongoDB? comme système de gestion de base de données mais si un intégrateur veut utiliser une autre système comme \verb?Mysql? alors il lui suffit de alors d'écrive son propre \verb?open-orchestra-model-bundle? qui utilise biensûr les différentes interfaces de \verb?open-orchestra-model-interface? mais adapté a \verb?Mysql?.