<<<<<<< HEAD
<<<<<<< HEAD
\documentclass[a4paper, 12pt]{report}
=======
\documentclass[a4paper]{report}
>>>>>>> 88f87b3... init
=======
\documentclass[a4paper, 12pt]{report}
>>>>>>> f6baf06... add intro

%====================== PACKAGES ======================

\usepackage[french]{babel}
\usepackage[utf8x]{inputenc}
%pour gérer les positionnement d'images
\usepackage{float}
\usepackage{amsmath}
\usepackage{graphicx}
\usepackage[colorinlistoftodos]{todonotes}
\usepackage{url}
%pour les informations sur un document compilé en PDF et les liens externes / internes
\usepackage{hyperref}
%pour la mise en page des tableaux
\usepackage{array}
\usepackage{tabularx}
%pour utiliser \floatbarrier
%\usepackage{placeins}
%\usepackage{floatrow}
%espacement entre les lignes
\usepackage{setspace}
\usepackage{lscape}
%modifier la mise en page de l'abstract
\usepackage{abstract}
%police et mise en page (marges) du document
\usepackage[T1]{fontenc}
\usepackage[top=2cm, bottom=2cm, left=2cm, right=2cm]{geometry}
%Pour les galerie d'images
\usepackage{subfig}

%====================== INFORMATION ET REGLES ======================

\setcounter{secnumdepth}{3}
\setcounter{tocdepth}{3}

\hypersetup{                            % Information sur le document
<<<<<<< HEAD
<<<<<<< HEAD
=======
>>>>>>> f6baf06... add intro
pdfauthor = {Amaury Lavielle},          % Auteurs
pdftitle = {Open Orchestra -
            CMS Symfony2},           % Titre du document
pdfsubject = {Rapport de stage},       % Sujet
pdfkeywords = {Symfony2, CMS},  % Mots-clefs
<<<<<<< HEAD
=======
pdfauthor = {Amaury Lavielle,
            Antoine Lelaisant},          % Auteurs
pdftitle = {Sydonie -
            Création d'un bundle Symfony2 gérant des documents selon la norme FRBR},           % Titre du document
pdfsubject = {Rapport de projet},       % Sujet
pdfkeywords = {FRBR, Sydonie, Document, Gestion de documents},  % Mots-clefs
>>>>>>> 88f87b3... init
=======
>>>>>>> f6baf06... add intro
pdfstartview={FitH}}                    % ajuste la page à la largueur de l'écran

%======================== DEBUT DU DOCUMENT ========================

\begin{document}

%régler l'espacement entre les lignes
\newcommand{\HRule}{\rule{\linewidth}{0.5mm}}

%page de garde
%%%%%%%%%%%%%%%%%%%%%%%%%%%%%%%%%%%%%%%%%
% University Assignment Title Page
% LaTeX Template
% Version 1.0 (27/12/12)
%
% This template has been downloaded from:
% http://www.LaTeXTemplates.com
%
% Original author:
% WikiBooks (http://en.wikibooks.org/wiki/LaTeX/Title_Creation)
%
% License:
% CC BY-NC-SA 3.0 (http://creativecommons.org/licenses/by-nc-sa/3.0/)
%
% Instructions for using this template:
% This title page is capable of being compiled as is. This is not useful for
% including it in another document. To do this, you have two options:
%
% 1) Copy/paste everything between \begin{document} and \end{document}
% starting at \begin{titlepage} and paste this into another LaTeX file where you
% want your title page.
% OR
% 2) Remove everything outside the \begin{titlepage} and \end{titlepage} and
% move this file to the same directory as the LaTeX file you wish to add it to.
% Then add \input{./title_page_1.tex} to your LaTeX file where you want your
% title page.
%
%%%%%%%%%%%%%%%%%%%%%%%%%%%%%%%%%%%%%%%%%

%----------------------------------------------------------------------------------------
%	PACKAGES AND OTHER DOCUMENT CONFIGURATIONS
%----------------------------------------------------------------------------------------
\begin{titlepage}

% \newcommand{\HRule}{\rule{\linewidth}{0.5mm}} % Defines a new command for the horizontal lines, change thickness here

\center % Center everything on the page

%----------------------------------------------------------------------------------------
%	HEADING SECTIONS
%----------------------------------------------------------------------------------------

\textsc{\LARGE Université de Caen Basse-Normandie}\\[1.5cm] % Name of your university/college
<<<<<<< HEAD
<<<<<<< HEAD
\textsc{\Large Rapport de stage}\\[0.5cm] % Major heading such as course name
\textsc{1 avril 2015 - 30 septembre 2015}\\[0.5cm] % Minor heading such as course title
=======
\textsc{\Large Rapport de projet}\\[0.5cm] % Major heading such as course name
\textsc{\large Projet annuel}\\[0.5cm] % Minor heading such as course title
>>>>>>> 88f87b3... init
=======
\textsc{\Large Rapport de stage}\\[0.5cm] % Major heading such as course name
\textsc{1 avril 2015 - 30 septembre 2015}\\[0.5cm] % Minor heading such as course title
>>>>>>> f6baf06... add intro

%----------------------------------------------------------------------------------------
%	TITLE SECTION
%----------------------------------------------------------------------------------------

\HRule \\[0.4cm]
<<<<<<< HEAD
<<<<<<< HEAD
{ \huge \bfseries Open orchestra}\\[0.4cm] % Title of your document
=======
{ \huge \bfseries Sydonie}\\[0.4cm] % Title of your document
>>>>>>> 88f87b3... init
=======
{ \huge \bfseries Open orchestra}\\[0.4cm] % Title of your document
>>>>>>> f6baf06... add intro
\HRule \\[1.5cm]

%----------------------------------------------------------------------------------------
%	AUTHOR SECTION
%----------------------------------------------------------------------------------------

\begin{minipage}{0.4\textwidth}
\begin{flushleft} \large
\emph{Auteurs:}\\
<<<<<<< HEAD
<<<<<<< HEAD
Amaury \textsc{Lavieille}
=======
Amaury \textsc{Lavieille}\\ % Your name
Antoine \textsc{Lelaisant}
>>>>>>> 88f87b3... init
=======
Amaury \textsc{Lavieille}
>>>>>>> f6baf06... add intro
\end{flushleft}
\end{minipage}
~
\begin{minipage}{0.4\textwidth}
\begin{flushright} \large
\emph{Enseignants:} \\
<<<<<<< HEAD
<<<<<<< HEAD
François \textsc{Rioult}
=======
Jean-Marc \textsc{Lecarpentier}\\
Hérvé \textsc{Lecrosnier}
>>>>>>> 88f87b3... init
=======
François \textsc{Rioult}
>>>>>>> f6baf06... add intro
\end{flushright}
\end{minipage}\\[4cm]

%----------------------------------------------------------------------------------------
%	DATE SECTION
%----------------------------------------------------------------------------------------

{\large \today}\\[3cm] % Date, change the \today to a set date if you want to be precise

%----------------------------------------------------------------------------------------
%	LOGO SECTION
%----------------------------------------------------------------------------------------

% Include a department/university logo - this will require the graphicx package

\begin{figure}[H]
    \begin{center}
      \includegraphics[scale=1]{images/LogoUnicaen}
    \end{center}
    \label{frbr-resume}
\end{figure}

%----------------------------------------------------------------------------------------

\vfill % Fill the rest of the page with whitespace

\end{titlepage}

%page blanche
~
%ne pas numéroter cette page
\thispagestyle{empty}

<<<<<<< HEAD
<<<<<<< HEAD
% \renewcommand{\abstractname}{}

\begin{abstract}
Développé par une équipe, de l'agence digitale interakting du groupe Business \& Decision, selon les principes de la méthode agile Scrum.
\paragraph{}
Open Orchestra est un CMS open-source, multi-sites, multi-langue, multi-devices, réalisé avec le framework PHP symfony2, conçu pour être hautement flexible et personalisable.
 En version bếta depuis un an, la première version est prévu pour le mois de septembre.
\paragraph{}
\paragraph{}
Developed by a team of the digital agency Interakting of Business \& Decision group, according to the agile Scum method.
\paragraph{}
Open Orchestra is a open-source CMS, multi-sites, multi-languages, multi-devices, developed with the PHP framework Symfony2, designed to be highly adaptable and expandable.
In beta since one year, the first version is planned for september.
\end{abstract}

\chapter*{Remerciements}
\addcontentsline{toc}{chapter}{Remerciements}
Tout d'abord, je tiens à remercier l'ensemble de l'agence Interakting pour m'avoir offert l'opportunité de réaliser mon stage au sein de l'agence.
\paragraph{}
Je tiens aussi à remercier plus particulièrement les différents membres de l'équipe d'Open Orchestra présent durant mon stage, pour tous leur  leur accueil au sein de l'équipe et leurs conseils

\begin{itemize}
\item[]
\item Nicolas Bouquet, Product Owner
\item[]
\item Nicolas Thal, Scrum master
\item[]
\item Nicolas Anne 
\item[]
\item Noël Gilain
\item[]
\item Romain Boëssel
\item[]
\end{itemize}
=======
 \renewcommand{\abstractname}{}

\begin{abstract}
Développé par une équipe, de l'agence digitale interakting du groupe Business \& Decision, selon les principes de la méthode agile Scrum.
\paragraph{}
Open Orchestra est un CMS open-source, multi-sites, multi-langue, multi-devices, réalisé avec le framework PHP symfony2, conçu pour être hautement flexible et personalisable.
 En version bếta depuis un an, la première version est prévu pour le mois de septembre.
\paragraph{}
\paragraph{}
Developed by a team of the digital agency Interakting of Business \& Decision group, according to the agile Scum method.
\paragraph{}
Open Orchestra is a open-source CMS, multi-sites, multi-languages, multi-devices, developed with the PHP framework Symfony2, designed to be highly adaptable and expandable.
In beta since one year, the first version is planned for september.
\end{abstract}


>>>>>>> 88f87b3... init
=======
% \renewcommand{\abstractname}{}

\begin{abstract}
Développé par une équipe, de l'agence digitale interakting du groupe Business \& Decision, selon les principes de la méthode agile Scrum.
\paragraph{}
Open Orchestra est un CMS open-source, multi-sites, multi-langue, multi-devices, réalisé avec le framework PHP symfony2, conçu pour être hautement flexible et personalisable.
 En version bếta depuis un an, la première version est prévu pour le mois de septembre.
\paragraph{}
\paragraph{}
Developed by a team of the digital agency Interakting of Business \& Decision group, according to the agile Scum method.
\paragraph{}
Open Orchestra is a open-source CMS, multi-sites, multi-languages, multi-devices, developed with the PHP framework Symfony2, designed to be highly adaptable and expandable.
In beta since one year, the first version is planned for september.
\end{abstract}

\chapter*{Remerciements}
\addcontentsline{toc}{chapter}{Remerciements}
Tout d'abord, je tiens à remercier l'ensemble de l'agence Interakting pour m'avoir offert l'opportunité de réaliser mon stage au sein de l'agence.
\paragraph{}
Je tiens aussi à remercier plus particulièrement les différents membres de l'équipe d'Open Orchestra présent durant mon stage, pour tous leur  leur accueil au sein de l'équipe et leurs conseils

\begin{itemize}
\item[]
\item Nicolas Bouquet, Product Owner
\item[]
\item Nicolas Thal, Scrum master
\item[]
\item Nicolas Anne 
\item[]
\item Noël Gilain
\item[]
\item Romain Boëssel
\item[]
\end{itemize}
>>>>>>> f6baf06... add intro
\tableofcontents
\thispagestyle{empty}
\setcounter{page}{0}
%ne pas numéroter le sommaire

%espacement entre les lignes d'un tableau
\renewcommand{\arraystretch}{1.5}

%====================== INCLUSION DES PARTIES ======================

~
\thispagestyle{empty}
%recommencer la numérotation des pages à "1"
\setcounter{page}{0}
\newpage
<<<<<<< HEAD
<<<<<<< HEAD
=======
>>>>>>> f6baf06... add intro
<<<<<<< HEAD
<<<<<<< HEAD
=======
>>>>>>> f6baf06... add intro
\chapter*{Introduction}
\addcontentsline{toc}{chapter}{Introduction}
        \paragraph{}
         Dans le cadre de ma deuxième année du master DNR2I (Master document numérique en réseau, ingénierie de l'Internet), j'ai effectué un stage chez Business \& Decision au sein de l'agence Interakting à Caen.
         \paragraph{}
          Pour une durée de six mois, du 1\up{er} Avril au 30 Septembre 2015, j'ai étais intégré au sein de l'équipe de développement du projet Open Orchestra.
         \paragraph{}
         Dans un premier temps, je vais vous présenter succinctement l'entreprise. Ensuite, je vous décrirais le projet Open Orchestra, l'origine de celui-ci, ces particularités, etc . Enfin, je vous parlerais de l'organisation et des méthodes mises en place pour la réalisation du projet.
<<<<<<< HEAD
=======
\chapter{Introduction}
    \section{Sujet du projet}
        \paragraph{}
        Ce projet vise à proposer une alternative à l'implémentation actuelle du framework PHP \og{}Sydonie\fg{} en le considérant non plus comme un framework à part entière, mais comme un bundle du framework Symfony2.

    \section{Contexte}
        \subsection{Les origines de Sydonie}
        \paragraph{}
        Le framework \og{}Sydonie\fg{} est issu d'un projet de recherche universitaire réalisé au sein du laboratoire du GREYC\footnote{Groupe de Recherche en Informatique, Image, Automatique et Instrumentation de Caen} de l'Université de Caen Basse-Normandie et a été initié par les deux tuteurs de ce projet: Jean-Marc \textsc{Lecarpentier} et Hervé \textsc{Lecrosnier}. Il propose un modèle permettant de manipuler des documents numériques plus efficacement en se basant sur le modèle FRBR\footnote{Functional Requirements for Bibliographic Records / Fonctionnalités requises des notices bibliographiques} défini par l'IFLA\footnote{The International Federation of Library Associations and Institutions}. Ce modèle apporte notamment une solution élégante aux problèmes liés à l'internationalisation des sites internet en s'appuyant sur une sémantique mieux adaptée que les systèmes \og{}clé-valeur\fg{} que l'on peut habituellement retrouver dans le paysage des frameworks et CMS liés au développement de sites web.


        \subsection{Les inconvénients}
        \paragraph{}
        Malgré les innovations et les nombreux avantages qu'il propose, Sydonie présente également des inconvénients non négligeables.

        \paragraph{}
        Tout d'abord, son développement a débuté il y a plusieurs années et a été mené par plusieurs développeurs qui ne sont plus, aujourd'hui, tous membres de l'équipe de développement. De plus, il y réside de larges parties de code qui n'ont pas systématiquement été documentées au fil des années et qui peuvent donc paraître un peu floues aujourd'hui. Ces deux facteurs rendent donc la maintenabilité du projet relativement pénible et fastidieuse.

        \paragraph{}
        Ensuite, bien que le code source du framework ait récemment été  revu intégralement afin d'y intégrer les namespace PHP, il reste encore de nombreuses fonctionnalités à développer telles que la gestion des routes ou des formulaires par exemple. Pour les chercheurs, il est donc dommageable de devoir passer du temps à développer des fonctionnalités de bas niveaux telles que celles-ci, alors que la plupart d'entre elles ont déjà été étudiées et implémentées dans d'autres frameworks.

        \paragraph{}
        Enfin, l'installation et l'initialisation d'un projet avec Sydonie est complexe. Il faut, en effet, télécharger quatre archives différentes, procéder à la configuration du framework et bien souvent adapter la configuration de la machine sur laquelle on souhaite travailler afin que l'application fonctionne correctement . De plus, la prise en main et la compréhension du modèle utilisé n'est pas si simple qu'il n'y paraît puisqu'il diffère complètement de ce que l'on peut trouver dans les autres frameworks. Il est donc à craindre que ces barrières ne découragent les futurs potentiels développeurs souhaitant travailler avec Sydonie.

    \section{Enjeux}
        \paragraph{}
        L'enjeu majeur de ce projet est de chercher à savoir si une implémentation différente du framework Sydonie est possible sous la forme d'un module de Symfony2. Ce module devra alors permettre aux chercheurs de poursuivre leurs travaux de recherche et de travailler directement sur des fonctionnalités du modèle proposé par Sydonie. De plus, il devra être assez documenté et simple d'utilisation pour que de potentiels développeurs puissent s'y intéresser, l'utiliser et le faire évoluer.

    \section{Objectifs}
        \paragraph{}
        Au vu des enjeux évoqués précédemment, l'objectif de ce projet va principalement être de prouver qu'une telle implémentation est possible. Il ne s'agit pas ici d'être exhaustifs et de reprendre l'intégralité des fonctionnalités de Sydonie, mais d'en extraire l'essentiel permettant d'exploiter les normes fondamentales décrites dans modèle FRBR.

        \paragraph{}
        De plus, afin de démontrer la pertinence d'un tel modèle sur le web, et notamment en ce qui concerne l'internationalisation des sites internet, il sera appréciable d'implémenter un système de négociation de contenu permettant de déterminer la version d'un document la plus pertinente pour un utilisateur donné.
>>>>>>> 88f87b3... init
=======
>>>>>>> f6baf06... add intro

\part{L'entreprise}
\chapter{Présentation de l'entreprise}
\section{Business \& Decision}

        Business \& Decision est un groupe international spécialisé dans trois grand domaines, la Business Intelligence\footnote{La Business Intelligence ou l'informatique décisionnelle représente les différentes solutions informatiques qui permettent l'exploitation des données d'une entreprise dans le but de faciliter la prise de décisions} (BI), la gestion de la relation client et enfin le e-business. 
\paragraph{}
        Le groupe a était créé en 1992 par Patrick Bensabat, aujourd'hui il est présent dans 15 pays et emploie plus de 2500 personnes.
\paragraph{}
       	  Événement majeurs et évolution du groupe depuis 1992 : 

       	  \begin{description}
       	  \item[]
       	  \item[1992] : Création de Business \& Decision par Patrick Bensabat autour de projets de Business Intelligence
       	  \item[1993-1996] : Mise en place des premiers Datawarehouses\footnote{Les Datawarehouses ou entrepôt de données désigne une base de données utilisée pour stocker et structurer des données de production d'une entreprise afin de fournir des informations stratégiques pour la prise de décisions} et applications de prévisions financières
       	  \item[1999-2000] : Création de la division CRM (Gestion de la relation client)
       	  \item[2002-2003] : Acquisition en Grande-Bretagne et au Benelux. Création d'agences régionales en France
       	  \item[2004] : Acquisition en Suisse et aux Pays-bas et implantation en Tunisie
       	  \item[2005] : Implantation aux Etats-Unis et acquisition de Metaphora \footnote{Métaphora est une société de service spécialisée dans la conduite du changement. Elle intervient dans toutes les étapes d’un projet pour faciliter l'appropriation du futur système d’informations par les utilisateurs finaux.}
       	  \item[2007-2008] : Création de la marque Interkating
       	  \item[2011] : Déploiement d'offres Cloud et Mobilité. Inauguration du Datacenter éco-responsable d'Eolas
       	  \item[2013] : Lancement de Datalyse, programme de recherche en Big Data. Implantation au Pérou
       	  \item[2014] : Création de Herewecan, agence de communication digitale, Lancement des pôles d'expertise Big Data, Transformation et Hub Mobile )
       	  \item[]       	  
       	  \end{description}
       	  
\section{Interakting}

        \paragraph{}
        Interakting est l'agence web du groupe B\&D  composée de 340 employés répartis principalement entre Caen et Paris, les équipes de l'agence interviennent sur des projets web  de grande envergure (Canal +, Bnp Paribas, Inra, Moët Hennessy, PSA Peugeot Citroën, ...).
        \paragraph{}
        Les différents projets de l'agence sont développés avec différents outils : eZ publish (système de gestion de contenu créé par l'entreprise eZ Systems As), PHP Factory (plateforme réalisée conjointement par Interakting et Zend Technologies) et de plus en plus avec Symfony2 (framework PHP écrit par SensionLabs) avec notamment Open Orchestra.

        
        
        
\part{Open Orchestra}
\chapter{Open Orchestra}
\section{Présentation}
        \paragraph{}
        Open Orchestra est un CMS (Content Management  System) open source en développement depuis un peu plus d'un an. Il est actuellement en version bêta, la sortie de la première version est prévu pour le mois de septembre.
        \paragraph{}
        Il est basé sur le framework PHP MVC Symfony 2 et MongoDB. Open Orchestra offre bien-sûr les fonctionnalités attendues par tous CMS : gestion de contenu, médiathèque, contribution de page, gestion de version, utilisateurs, worklow, rôles,  multi-site, multi- langue,  multi-device , etc.
        \paragraph{}
        La grande particularité d'Open Orchestra est son faible couplage au niveau du code, mais aussi au niveau des solutions techniques, c'est-à-dire que l'on peut facilement remplacer ou étendre un de ses composants. Je reviendrais plus en détail sur ce point dans une section ultérieure.
        
        \section{Architecture}
       Avec Open Orchestra le \og Back-office \fg{} et le \og Front office \fg{} peuvent être dissociés dans deux applications Symfony différentes. 

        Ce découpage a de nombreux avantages, tout d'abord un seul \og Back-office \fg{}  peut gérer plusieurs sites (\og Front-office \fg{}), de plus cela permet de mettre le \og Back-office \fg{} et le \og Front-office \fg{} sur des serveurs différents, l'unique condition est qu'ils doivent tous utiliser la même base de données.
        \paragraph{}
		Pour finir, Open Orchestra utilise une API RESTFull\footnote{REST (Representational State Transfer) est un style d'architecture qui défini différentes réglés (ressource identifié unitairement, utilisation des verbes HTTP :  POST, DELETE, PUT, GET) pour accéder et manipuler des ressources } pour  accéder, gérer ces différentes entités (pages, utilisateurs, contenus, etc).
	    \paragraph{}	
		Le schéma~\ref{architecture} présente l'architecture d'Open Orchestra avec un \og Back-office \fg{} qui gère plusieurs \og Fronts \fg{}.
		\begin{figure}[H]
        \begin{center}
          \includegraphics[scale=0.75]{images/architecture_open_orchestra}
        \end{center}
        \caption{Exemple d'architecture d'utilisation d'Open Orchestra}
        \label{architecture}
      \end{figure}

   
\section{Fonctionnalités}
	    \paragraph{}
	    Dans ce chapitre je vais vous présenter les différentes fonctionnalités clés d'Open Orchestra pour permettre une bonne compréhension du projet.
	      \subsection{Blocs}
	        \label{Blocs}  
	       \paragraph{}
	      Dans Open Orchestra tous les éléments visibles en front sont représentés par des blocs. Un bloc est simplement une entité avec une liste d'attributs qui varient selon le type de bloc. Chaque type de bloc est indépendant, c'est-à-dire que eux seuls connaissent leurs attributs, leurs façon de s'afficher en front ou en back-office ou encore la façon dont ils peuvent être renseigné en back-office. 
	       \paragraph{}
	       Pour permettre cette indépendance, Open Orchestra utilise le design pattern stratégie. Le pattern stratégie permet de rendre une famille d'algorithmes interchangeables et ainsi  la possibilité d'exécuter un traitement spécifique selon le contexte.
	       	 \paragraph{}
	       	 Ce pattern est utilisé de nombreuse fois dans Open Orchestra, comme par exemple dans le cas de l'affichage d'un bloc.
          \paragraph{}
	       Comme le présente le diagramme de classe~\ref{pattern strategy} chaque bloc possède une classe qui indique la façon dont il doit s'afficher, pour cela la classe possède deux méthodes, la première qui indique le type de bloc qu'elle supporte (\verb?support?) et la méthode \verb?show? qui fournit le rendu, d'un autre coté, il y a une classe appelée \verb?manager?  qui est la seule a connaitre toutes les stratégies des différents blocs. 
		\begin{figure}[H]
        \begin{center}
          \includegraphics[scale=0.75]{images/strategy_block}
        \end{center}
        \caption{Utilisation du pattern stratégie pour l'affichage des bloc en front}
        \label{pattern strategy}
      \end{figure}
         \paragraph{}
	       Ainsi, lorsque que l'on désire afficher un bloc, il suffit de demander au manager d'afficher ce bloc, ce dernier va chercher parmi les stratégies qu'il connait celle qui supporte (méthode \verb?support?) le bloc et s'il en trouve une alors il exécute la méthode (\verb?show?) de la stratégie.
	        
	      	\paragraph{}
	      	Par défaut, le CMS propose de nombreux types blocs (annexe ~\ref{annexe_bloc}) comme par exemple un bloc pour lister un type de contenu, afficher un texte formaté, afficher un carrousel ou un média, un menu, une carte, un bloc de contact, etc.
	      \paragraph{}
	      	 Bien-sûr, il est possible pour un intégrateur \footnote{Dans le cadre de ce rapport un intégrateur indique une personne ou une équipe qui utilise Open Orchestra afin de  l'étendre pour des besoins particuliers}, de créer son propre type de bloc, il lui suffit de développer les différentes stratégies (affichage en front et back-office, formulaire en back-office, etc) nécessaires à un bloc.  
         \subsection{Nodes}
         \paragraph{}
         Un des points central d'un CMS est les pages. Sur Open Orchestra les pages sont identifiées comme des noeuds (\og nodes \fg{}).
          Les nodes sont simplement des conteneurs qui contiennent des zones. Les zones quant à elles permettent d'organiser la page, elles peuvent contenir des sous-zones ou des blocs (exemple en annexe~\ref{annexe_node}) ce qui permet un découpage fin de la page. 
         \paragraph{}
         Il existe trois types de nodes : 
         \begin{itemize}
         \item[]
         \item  Les nodes qui représentent les pages visibles en \og front \fg{}.
          \item[]
         \item  Les \og node tranverse \fg{} qui contiennent les blocs transverses, c'est-à-dire les blocs qui sont communs à plusieurs pages comme un bloc \og menu \fg{} ou encore un bloc \og footer \fg{}.
          \item[]
         \item Les nodes qui permettent de contribuer les pages d'erreurs (404, 503) d'un site.
         \item[]
         \end{itemize}
         \paragraph{}
		Le schéma~\ref{node} montre un exemple typique de page avec trois zones, le header qui contient un bloc menu, le footer un bloc footer et une zone principale découpée elle-même en deux zones avec un bloc de contact pour la zone de droite et un bloc de texte à gauche.
		\begin{figure}[H]
        \begin{center}
          \includegraphics[scale=0.75]{images/node}
        \end{center}
        \caption{Exemple de page}
        \label{node}
      \end{figure}
         \subsection{Content type}
         Le second point important d'un CMS est les contenus, Open Orchestra est conçu pour faciliter la création de type de contenu (actualité, client, etc, ...).
          \paragraph{}
          En effet, Open Orchestra propose une approche graphique avec un formulaire pour créer ou éditer des types de contenus comme l'illustre la figure~\ref{content_type}.
             		\begin{figure}[H]
          \includegraphics[scale=0.6]{images/content_type}
        \caption{Formulaire d'édition du type de contenu news}
        \label{content_type}
      \end{figure}
          Un type de contenu est composé de différents champs avec différentes options, par exemple pour un champ de type texte il peut y avoir une option pour limiter le nombre de caractère ou encore une valeur par défaut.
          \paragraph{}
          Par défaut, Open Orchestra offre une liste de types de champs (date, texte, monnaie, média, email, entier, zone de texte riche, etc) qui comme pour tous les composants d'Open Orchestra peut être étendu par d'autre type de champ personnalisé. 
\section{Caractéristiques}
   \subsection{Performance}
   Open Orchestra a été développé pour supporter une charge de trafic importante. Pour cela le CMS exploite au niveau du front l'ESI (Edge Side Includes) couplé à un reverse proxy\footnote{Un reverse proxy est un serveur qui traite les requêtes en amont du serveur web. L'intérêt d'un reverse proxy est multiple gestion de cache, chiffrement, répartition de charge}.
   \paragraph{}
   L'ESI est un langage de balisage HTML qui permet de diviser une page en différents éléments dont les rendus sont faits dans différentes requêtes par le serveur web. Cela permet d'avoir un cache HTTP sur les différents éléments et ainsi rafraichir seulement les éléments obsolètes de la page et non toute la page.
   \paragraph{}
   Comme nous l'avons vu dans la section~\ref{Blocs} les pages sont découpées en différents blocs ainsi l'utilisation de l'ESI est adapté et permet une amélioration significative des performances.
   \paragraph{}
   Si nous reprenons l'exemple de notre page (schéma~\ref{node}), le schéma~\ref{esi} présente le processus effectué par le serveur dans le cas où les blocs \verb?header? et \verb?footer? sont déjà en cache lorsqu'un utilisateur demande la page.
   \paragraph{}
   		\begin{figure}[H]
        \begin{center}
          \includegraphics[scale=0.75]{images/esi}
        \end{center}
        \caption{Processus d'affichage d'une page qui utile l'ESI}
        \label{esi}
      \end{figure}
   
   \subsection{Modularités}
   Comme je vous l'expliquais en introduction la principale particularité d'Open Orchestra est sa modularité. Pour cela, les différents composants du CMS sont répartis dans différents bundles (cf section ~\ref{bundle}).
   \paragraph{}
   \begin{itemize}
   \item \textbf{open-orchestra-base-bundle} : Classes communes au back-office et au front-office
            \item[]
   \item \textbf{open-orchestra-base-api-bundle} : Api d'Open Orchestra
         \item[]   
   \item \textbf{open-orchestra-base-api-mongo-model-bundle} : Implémentation des models nécessaires à l'api pour \verb?MongoDB? 
            \item[]
   \item \textbf{open-orchestra-cms-bundle} : Logique du back-office
            \item[]
   \item \textbf{open-orchestra-front-bundle} : Logique du front-office
            \item[]
   \item \textbf{open-orchestra-display-bundle} : Logique d'affichage des blocs en Front-office
            \item[]
   \item \textbf{open-orchestra-media-bundle} : Médiathèque d'Open Orchestra
            \item[]
   \item \textbf{open-orchestra-media-admin-bundle} : Administration de la médiathèque d'Open Orchestra
            \item[]
   \item \textbf{open-orchestra-model-interface} : Interfaces des models utilisés par les autres bundles 
            \item[]
   \item \textbf{open-orchestra-model-bundle} : Implémentation des models pour \verb?MongoDB? 
            \item[]
   \item \textbf{open-orchestra-user-bundle} : Gestion des utilisateurs 
            \item[]
   \item \textbf{open-orchestra-workflow-function-bundle} : Système de workflow pour le back-office 
   \end{itemize}
   \paragraph{}
    Une application front-office n'a aucun intérêt à charger toute la logique du back-office ainsi grâce à ce découpage chaque application (front-office ou back-office) charge uniquement les composants qui lui est nécessaire.
   \paragraph{}
   De plus, cela permet de désactiver facilement une fonctionnalités. Par exemple, si un intégrateur n'a pas bessoin de gérer des médias il peut alors désactiver la médiathèque en n'utilisant pas les bundles (\verb?open-orchestra-media-bundle? et \verb?open-orchestra-media-admin-bundle?).

   \paragraph{}
   Un autre avantage est la facilité de remplacement d'un composants. Par défaut Open Orchestra utilise \verb?MongoDB? comme système de gestion de base de données, mais si un intégrateur veut utiliser un autre système comme \verb?Mysql? alors il lui suffit d'écrire son propre \newline  \verb?open-orchestra-model-bundle? qui utilise bien-sûr les différentes interfaces de  \newline  \verb?open-orchestra-model-interface? adaptés a \verb?Mysql?.
\part{Organisation du développement}
<<<<<<< HEAD
=======

\part{Présentation}
    <<<<<<< HEAD
<<<<<<< HEAD
=======
>>>>>>> f6baf06... add intro
\chapter*{Introduction}
\addcontentsline{toc}{chapter}{Introduction}
        \paragraph{}
         Dans le cadre de ma deuxième année du master DNR2I (Master document numérique en réseau, ingénierie de l'Internet), j'ai effectué un stage chez Business \& Decision au sein de l'agence Interakting à Caen.
         \paragraph{}
          Pour une durée de six mois, du 1\up{er} Avril au 30 Septembre 2015, j'ai étais intégré au sein de l'équipe de développement du projet Open Orchestra.
         \paragraph{}
         Dans un premier temps, je vais vous présenter succinctement l'entreprise. Ensuite, je vous décrirais le projet Open Orchestra, l'origine de celui-ci, ces particularités, etc . Enfin, je vous parlerais de l'organisation et des méthodes mises en place pour la réalisation du projet.
<<<<<<< HEAD
=======
\chapter{Introduction}
    \section{Sujet du projet}
        \paragraph{}
        Ce projet vise à proposer une alternative à l'implémentation actuelle du framework PHP \og{}Sydonie\fg{} en le considérant non plus comme un framework à part entière, mais comme un bundle du framework Symfony2.

    \section{Contexte}
        \subsection{Les origines de Sydonie}
        \paragraph{}
        Le framework \og{}Sydonie\fg{} est issu d'un projet de recherche universitaire réalisé au sein du laboratoire du GREYC\footnote{Groupe de Recherche en Informatique, Image, Automatique et Instrumentation de Caen} de l'Université de Caen Basse-Normandie et a été initié par les deux tuteurs de ce projet: Jean-Marc \textsc{Lecarpentier} et Hervé \textsc{Lecrosnier}. Il propose un modèle permettant de manipuler des documents numériques plus efficacement en se basant sur le modèle FRBR\footnote{Functional Requirements for Bibliographic Records / Fonctionnalités requises des notices bibliographiques} défini par l'IFLA\footnote{The International Federation of Library Associations and Institutions}. Ce modèle apporte notamment une solution élégante aux problèmes liés à l'internationalisation des sites internet en s'appuyant sur une sémantique mieux adaptée que les systèmes \og{}clé-valeur\fg{} que l'on peut habituellement retrouver dans le paysage des frameworks et CMS liés au développement de sites web.


        \subsection{Les inconvénients}
        \paragraph{}
        Malgré les innovations et les nombreux avantages qu'il propose, Sydonie présente également des inconvénients non négligeables.

        \paragraph{}
        Tout d'abord, son développement a débuté il y a plusieurs années et a été mené par plusieurs développeurs qui ne sont plus, aujourd'hui, tous membres de l'équipe de développement. De plus, il y réside de larges parties de code qui n'ont pas systématiquement été documentées au fil des années et qui peuvent donc paraître un peu floues aujourd'hui. Ces deux facteurs rendent donc la maintenabilité du projet relativement pénible et fastidieuse.

        \paragraph{}
        Ensuite, bien que le code source du framework ait récemment été  revu intégralement afin d'y intégrer les namespace PHP, il reste encore de nombreuses fonctionnalités à développer telles que la gestion des routes ou des formulaires par exemple. Pour les chercheurs, il est donc dommageable de devoir passer du temps à développer des fonctionnalités de bas niveaux telles que celles-ci, alors que la plupart d'entre elles ont déjà été étudiées et implémentées dans d'autres frameworks.

        \paragraph{}
        Enfin, l'installation et l'initialisation d'un projet avec Sydonie est complexe. Il faut, en effet, télécharger quatre archives différentes, procéder à la configuration du framework et bien souvent adapter la configuration de la machine sur laquelle on souhaite travailler afin que l'application fonctionne correctement . De plus, la prise en main et la compréhension du modèle utilisé n'est pas si simple qu'il n'y paraît puisqu'il diffère complètement de ce que l'on peut trouver dans les autres frameworks. Il est donc à craindre que ces barrières ne découragent les futurs potentiels développeurs souhaitant travailler avec Sydonie.

    \section{Enjeux}
        \paragraph{}
        L'enjeu majeur de ce projet est de chercher à savoir si une implémentation différente du framework Sydonie est possible sous la forme d'un module de Symfony2. Ce module devra alors permettre aux chercheurs de poursuivre leurs travaux de recherche et de travailler directement sur des fonctionnalités du modèle proposé par Sydonie. De plus, il devra être assez documenté et simple d'utilisation pour que de potentiels développeurs puissent s'y intéresser, l'utiliser et le faire évoluer.

    \section{Objectifs}
        \paragraph{}
        Au vu des enjeux évoqués précédemment, l'objectif de ce projet va principalement être de prouver qu'une telle implémentation est possible. Il ne s'agit pas ici d'être exhaustifs et de reprendre l'intégralité des fonctionnalités de Sydonie, mais d'en extraire l'essentiel permettant d'exploiter les normes fondamentales décrites dans modèle FRBR.

        \paragraph{}
        De plus, afin de démontrer la pertinence d'un tel modèle sur le web, et notamment en ce qui concerne l'internationalisation des sites internet, il sera appréciable d'implémenter un système de négociation de contenu permettant de déterminer la version d'un document la plus pertinente pour un utilisateur donné.
>>>>>>> 88f87b3... init
=======
>>>>>>> f6baf06... add intro


    \input{./sydonie.tex}

    \input{./symfony2.tex}

\part{Implémentation}
    \input{./model-implementation.tex}

    \input{./generator-entity.tex}

    \input{./generator-crud.tex}

    \input{./content-negociation.tex}

    \input{./document-composite.tex}
\part{Conclusions et perspectives}
    \chapter*{Bilan}
\addcontentsline{toc}{chapter}{Bilan}
Durant le stage, ce fut un réel plaisir de travailler avec l'équipe du projet Open Orchestra et plus en général avec les différentes personnes d'Interakting que j'ai côtoyé à l'agence de Caen ou de Paris lors de mes déplacements.
\paragraph{}
J'ai été directement intégré au projet comme aurait pu l'être un autre développeur de l'agence ce qui a été très constructif pour moi et important de voir que l'on me faisait confiance.
\paragraph{}
Travailler sur l'élaboration d'un projet Open Source en mode agile d'un point de vue plus orienté éditeur que intégrateur (réalisation de site pour des clients) a été très intéressant et motivant.  De plus, cela m'a permis de voir et mettre en pratique la méthode agile SCRUM sur un projet concret.
\paragraph{}
D'un point de vue technique, cela fut très enrichissant, j'ai pu comprendre et approfondir de nombreux concepts avancés de Symfony2. Mais plus généralement, j'ai appris énormément sur les bonnes pratiques (test, psr2, principe SOLID \footnote{Les cinq principes de bases de la programmation orientée objet :
\begin{itemize}
\item Single responsibility : Une classe doit avoir une seule responsabilité
\item Open/closed : Une classe doit être ouverte à l'extension, mais fermé à la modification
\item Liskov Substitution : Une instance de type A doit pouvoir être remplacé par une instance de type B tel que B implémente A
\item Interface segregation : Préférer plusieurs interfaces spécifiques plutôt qu'une seule interface générale
\item Dependency Inversion : Il ne faut pas dépendre des implémentations, mais des abstractions
\end{itemize}
}
) mais aussi découvrir et utiliser de nouveaux outils/langages (Travis, Ansible, Vagrant, CoffeeScript, Jmeter). 
De plus, durant mon stage j'ai assisté à des formations sur différent sujet (Git, Travis, Scrutinizer) proposé tous les jeudis de 13h à 15h par des développeurs de l'entreprise.
\paragraph{}
Pour conclure, ce stage à largement répondu à mes attentes que ce soit sur le projet en lui-même ou sur les points techniques.

>>>>>>> 88f87b3... init
=======
>>>>>>> f6baf06... add intro
\input{./appendix.tex}

\newpage

%récupérer les citation avec "/footnotemark"
\nocite{*}

%choix du style de la biblio
\bibliographystyle{plain}
%inclusion de la biblio
\bibliography{bibliographie.bib}
%voir wiki pour plus d'information sur la syntaxe des entrées d'une bibliographie

\end{document}